% So we make this "beamer" rather than document!

\documentclass[11pt]{beamer}
% For handout add ,handout after 11pt

\usetheme[sectionpage=none,numbering=none]{metropolis}           % Use metropolis theme
	% To do printouts, add ", handout"  after aspectratio.
\usepackage{booktabs}
\usepackage{graphicx}
\usepackage{color}

\title{Backwards Design \\ in Data Science}
\author{\small Nick Eubank}
\date{\vspace*{.3in} \date}


% This is the beginning of a real document!
\begin{document}


\begin{frame}
\maketitle
\end{frame}

\begin{frame}[c]{Backwards Design}

Approach to planning data science projects
  \begin{itemize}
    \pause \item (Though backwards design isn't unique to DS)
  \end{itemize}
Goals:
\begin{itemize}
  \item Minimize wasted effort
  \pause \item Make sure you develop explicit goals
  \begin{itemize}
    \item Not get lost in your tools and data
  \end{itemize}
\end{itemize}
\end{frame}

\begin{frame}[c]{Backwards Design}

  Start with where you want to end up, then work backwards

\end{frame}

\begin{frame}[c]{Backwards Design}
  \begin{enumerate}
    \pause \item Determine Problem / Topic Area
    \pause \item What \emph{question} are you seeking to answer?
    \pause \item What does an answer to your question look like?
    \pause \item What variables do you need to generate that answer?
    \pause \item What data contains those variables?
  \end{enumerate}
\end{frame}

\begin{frame}[c]{Step 0: Define the \alert{Problem / Topic}}
Why are you doing this project? \\
\pause What \alert{motivates} your investigation?\\
\vspace*{0.1cm}
\pause Examples:
\begin{itemize}
  \item We don't know how to reduce mass incarceration
  \item My business can't identify potential customers
  \item We can't diagnose Alzheimers
\end{itemize}
\end{frame}


\begin{frame}[c]{Step 1: What \alert{question} are you seeking to answer?}
The tools of data science are fundamentally designed to \alert{answer questions}, \pause so to before you pick your tools, you have to decide \alert{what question you wish to answer.}\\
\vspace*{0.1cm}
\pause $\Rightarrow$ The MOST important part of your project\\
\end{frame}

\begin{frame}[c]{Step 1: What \alert{question} are you seeking to answer?}
Most important because:

\begin{itemize}
  \pause \item if you can't define the question you are seeking to answer, \alert{you'll find yourself lost in your data}, or worse
  \pause \item after finishing your project, you'll realizing the question you answered doesn't help solve the problem that motivated you.
\end{itemize}
\pause  $\Rightarrow$ Invest in this stage of your project \emph{before} you dive into the data!
\end{frame}


\begin{frame}[c]{Step 1: What \alert{question} are you seeking to answer?}

A critical feature of a good question is that it is \emph{tractable} and \emph{answerable} in a data science project. \\
\begin{itemize}
  \item If your question does not directly imply a course of action in your data science project, it's too vague.
\end{itemize}
\end{frame}

\begin{frame}[c]{Step 1: What \alert{question} are you seeking to answer?}
Not answerable:
\begin{itemize}
  \item What policies reduce mass incarceration?
  \item Can machine learning help me identify potential customers.
  \item What indicates Alzheimers?
\end{itemize}
\pause
Answerable:
\begin{itemize}
  \item Does the availability of grand juries result in longer sentences?
  \item What attributes are common to the customers who buy the most from my business?
  \item Are there lab results common to patients diagnosed (post-mortem) with Alzheimers not common to patients without Alzheimers?
\end{itemize}
\end{frame}

\begin{frame}[c]{Step 1: What \alert{question} are you seeking to answer?}
How do I know if my answer is answerable / tractable?

\begin{enumerate}
  \pause \item Can you hypothesize an answer to your question? \\
  i.e. Can you state what you think might be the answer to your question?
  \pause \item Can you imagine what the answer to your question looks like?
\end{enumerate}
\end{frame}

\begin{frame}[c]{Step 2: What does \alert{an answer} to your question look like?}
\alert{Write down} what the answer to your question will look like!
\pause
\begin{itemize}
  \item A figure
  \item A table or regression
  \item A dataset with predicted values
\end{itemize}
\pause
$\Rightarrow$ Ask yourself: if I gave that to my stakeholder / put it in a paper, would people be pleased?\\
\pause (OK, they might want robustness, and extensions, but at its core, is this an answer?)
\end{frame}

\begin{frame}[c]{Step 2: What does \alert{an answer} to your question look like?}
  \begin{itemize}
    \pause \item \textbf{Incarceration}: A regression that shows differences in sentences for arrestees in counties with standing grand juries as compared to counties without standing grand juries, controlling for details of charges.
    \pause \item \textbf{Business:} A table showing the performance of a machine learning model that predicts (past) customer behavior using pre-purchase data on customer website interactions (and model parameters).
    \pause \item \textbf{Alzheimers:} A regression showing a strong correlation between certain test results and receiving a positive diagnosis of Alzheimers in (post-mortem) testing.
  \end{itemize}
\end{frame}

\begin{frame}[c]{Step 2: What does \alert{an answer} to your question look like?}
But it's not enough to imagine \emph{one} answer. You should be able to imagine what an answer to your question looks like if your hypothesis \alert{is true} and the if your hypothesis \alert{is false}. \\
\pause Otherwise your question isn't falsifiable! \\
\vspace*{0.3cm}
Write down what your answer looks like if your hypothesis is true, \emph{and} if it's false!
\end{frame}

\begin{frame}[c]{Step 3: What do you need to generate that answer?}
  Congratulations! You've just specified the goal of your analysis! \\
  \pause In my view, that is actually the hardest part of being a good data scientist. \\
  \pause ...Though probably not the part that will take up the majority of your time.
\end{frame}

\begin{frame}[c]{Step 3: What do you need to generate that answer?}
  So you now have in mind a table you want to generate. What data and variables do you need to create that result?
  \pause
  \pause For each variable, specify:
  \begin{enumerate}
    \item What do you need the variable to measure?
    \item For what population do you need the variable defined?
  \end{enumerate}
\end{frame}

\begin{frame}[c]{Step 4: Where can you get those variables?}
  \begin{enumerate}
    \item Where can you get those variables?, and
    \item How will you relate your different datasets?
  \end{enumerate}
\end{frame}

\begin{frame}[c]{Is this what everyone does?}
  \pause Not that I'm aware of. \\
  \pause Most people who are \emph{successful} seem to do this implicitly \\
  \pause People who don't use this, in my experience, tend to flail.
\end{frame}

\begin{frame}[c]{Final Project}
  In teams of \emph{up to} three people, \pause you will have to develop \emph{your own project idea} from scratch using this model.
  \begin{itemize}
    \item Just as the last project emphasized all the data tasks \emph{before} analysis,
    \pause \item The goal of this is to emphasize all the things you do \emph{before} you touch your data!
  \end{itemize}
  \vspace*{0.1cm}
  If you're looking for a model... look back at the write up of your assignment for the mid-semester project!
\end{frame}

\begin{frame}[c]{In Class}
You have been approached by a campaign to reduce teen vaping.

Over the past year, they've tried several different pilot programs in several cities.

They just got a huge donation, and want to know what they should do with it.
\end{frame}

\end{document}
